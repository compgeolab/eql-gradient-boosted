The equivalent source technique is a powerful and widely used method for
processing gravity and magnetic data.  Nevertheless, its major
drawback is the large computational cost in terms of processing time and
computer memory.
We present two techniques for reducing the computational cost of equivalent
source processing: block-averaging source locations and the
gradient-boosted equivalent source algorithm.
Through block-averaging, we reduce the number of source coefficients that
must be estimated while retaining the minimum desired resolution in the final
processed data.
With the gradient boosting method, we estimate the sources coefficients in
small batches along overlapping windows, allowing us to reduce the computer
memory requirements arbitrarily to conform to the constraints of the
available hardware.
We show that the combination of block-averaging and gradient-boosted
equivalent sources is capable of producing accurate interpolations through
tests against synthetic data.
Moreover, we demonstrate the feasibility of our method by gridding a gravity
dataset covering Australia with over 1.7 million observations using a modest
personal computer.
