\documentclass[twocolumn]{article}
\usepackage[left=0.7in,right=0.7in,top=1in,bottom=1in]{geometry}
\setlength{\columnsep}{2\columnsep}
\usepackage[utf8]{inputenc}
\usepackage{amssymb}
\usepackage{amsmath}
\usepackage{graphicx}
\usepackage[round]{natbib}
\usepackage{url}
\usepackage[pdftex,colorlinks=true]{hyperref}
\usepackage{fancyhdr}


% Define title, authors, affiliations and DOI
% ===========================================
\newcommand{\Title}{A better strategy for interpolating gravity and magnetic data}
\newcommand{\Author}{
    S.R. Soler,
    L. Uieda
}
\newcommand{\AuthorAffil}{
    {\large
        S.R. Soler$^{1,2}$,
        L. Uieda$^{3}$
    }
    \\[0.4cm]
    {\small $^{1}$CONICET, Argentina (santiago.r.soler@gmail.com)} \\
    {\small $^{2}$Instituto Geofísico Sismológico Volponi, UNSJ, Argentina} \\
    {\small $^{3}$University of Liverpool} \\

}
\newcommand{\DOI}{doi:\href{https://doi.org/xxx.xxx/xxxxxx}{10.1093/xxx.xxx/xxxxxx}}
\newcommand{\DOILink}{\href{https://doi.org/xxx.xxx/xxxxxx}{doi.org/xxx.xxx/xxxxxx}}


% Configure header and hypersetup
% ===============================
\pagestyle{fancy}
\fancyhf{}
\lhead{
    \fontsize{9pt}{12pt}\selectfont
    \Author{}, 2019. \DOI{}
}
\rhead{\fontsize{9pt}{12pt}\selectfont \thepage}
\renewcommand{\headrulewidth}{0pt}
\hypersetup{
    allcolors=blue,
    pdftitle={\Title},
    pdfauthor={\Author},
}


%%%%%%%%%%%%%%%%%%%%%%%%%%%%%%%%%%%%%%%%%%%%%%%%%%%%%%%%%%%%%%%%%%%%%%%%%%%%%%%

\begin{document}

\title{\Title}
\author{\AuthorAffil}
\date{
    \normalsize
    \today
}
\maketitle

\begin{abstract}
    My abstract
    \\[0.5cm]
    \textbf{Keywords:}
    My keywords
\end{abstract}

%%%%%%%%%%%%%%%%%%%%%%%%%%%%%%%%%%%%%%%%%%%%%%%%%%%%%%%%%%%%%%%%%%%%%%%%%%%%%%%

\section{Introduction}

Potential field data, like gravity and magnetics, are useful for geophysics exploration.
The potential field data can be obtained from ground, airborne or satellite
measurements.
On ground surveys the data is often gathered on scattered points or following irregular
paths, usually at the topographic surface.
The data points have different heights.
On airborne surveys the data is gathered on flight paths, producing a large number of
data points concentrated along almost straight lines.
The height of measurement could change because of the vertical movement of the airship.
To improve visualization of the data for interpretation and to prepare the data for
processing through different methods (RTP, upward continuation, Euler deconvolution,
etc), the data needs to be interpolated on a regular grid of data points.

There are several methods for interpolating 2D data like minimum curvature, biharmonic
Splines or Kriging.
These all-purpose methods show a few problems when applied to potential field data.
(i) They don't take into account the height of the observation points. Potential field
data has a strong dependency on measurement height.
(ii) The predicted grid data is not necessary an Harmonic function.
(iii) They don't usually take into account the anisotropy of the observation points in
ground and airborne surveys.

One of the most widely used methods for interpolating gravity and magnetic data is the
equivalent sources technique (also known as equivalent layer, radial basis functions or
Green's functions interpolation).
It consists in estimating a source distribution that produces the same field as the one
measured and using this estimate to predict new values.
\citet{dampney1969} was the first to introduce the equivalent source technique.

The EQL can be used for interpolation of potential field data \citep{cordell1992,
cooper2000}.
It has been also used for applying a reduction to the pole to magnetic data
\citep{guspi2009, silva1986, emilia1973, nakatsuka2006} or an upward continuation
(cite).

The most widely used source distribution is a set of point sources.
The potential field of point sources is easy to compute.
\citet{barnes2011} make use of prisms.
How to locate these point masses?
Existing approaches:
(i) one point source beneath each observation point at a constant depth (cite),
(ii) one point source beneath each observation point at a depth proportional to the
distance to its nearest data points (cite),
(iii) a regular grid of sources \citep{barnes2011}.

We want to find out which distribution of sources produces the best interpolations.


%%%%%%%%%%%%%%%%%%%%%%%%%%%%%%%%%%%%%%%%%%%%%%%%%%%%%%%%%%%%%%%%%%%%%%%%%%%%%%%

\section{The equivalent sources technique}

Mathematical background of the EQL\@.
How it can be understood in terms of Green's functions problem.


%%%%%%%%%%%%%%%%%%%%%%%%%%%%%%%%%%%%%%%%%%%%%%%%%%%%%%%%%%%%%%%%%%%%%%%%%%%%%%%

\section{Source distributions}

Describe the proposed source distributions as combination of layouts and depths
strategies.

The choice of a source distribution is not trivial and plays an important role on the
accuracy of the predictions.

The most widely used source distributions are: a regular grid of point sources and
one point source beneath each observation point.

The \emph{regular grid} creates an homogeneous distribution of point sources bellow the
surveyed region.
It could create too many sources on areas without data and very few ones on
areas with a high number of observation points, leading to an underfitting of the
observed data.
This could be solved by reducing the grid spacing and therefore increasing the number of
point sources, heavily increasing the computational load of the interpolation process.

The \emph{source beneath data} layout adds sources where they are more needed.
If the survey is composed by a large number of data points clustered on paths
(like an airborne survey), putting one source beneath each observation point will create
an anisotropic source distribution: there will be several sources distributed along
a privileged direction.
This could introduce aliases on the predictions and increase the computational load for
fitting the predictor.

We propose a new source distribution that could simultaneously solve the problems of the
preceding ones: the \emph{block-median sources}.
It consist in dividing the region in blocks of equal size, computing the median location
of the observation points that fall inside each block, and putting one point source
bellow this block-median coordinate.
It creates one point source beneath each populated block, solving the problem of the
\emph{grid sources} layout, but also reducing the number of point sources in comparison
with the \emph{source beneath data} layout.

The depth of the point sources can be chosen following different criteria.
The most simple option is to locate all point sources at the same depth, which we will
call \emph{constant depth} in the future.
It could introduce some artifacts, like

\begin{table*}
    \begin{minipage}{80mm}
        \caption{
            Source distributions as combinations of source layouts and depth
            strategies.
        }
        \label{tab:source-distributions}
        \begin{tabular}{lccc}
            & Source bellow data & Block-median sources & Grid sources \\ \hline
            Constant depth          & $\checkmark$ & $\checkmark$ & $\checkmark$ \\
            Relative depth          & $\checkmark$ & $\checkmark$ & $\times$     \\
            Variable relative depth & $\checkmark$ & $\checkmark$ & $\times$     \\
        \end{tabular}
    \end{minipage}
\end{table*}


%%%%%%%%%%%%%%%%%%%%%%%%%%%%%%%%%%%%%%%%%%%%%%%%%%%%%%%%%%%%%%%%%%%%%%%%%%%%%%%

\section{Comparison of source distributions}

\subsection{Synthetic Model}

Describe forward model made out of prisms.
Show prisms model?

\subsection{Synthetic Surveys}

Describe the ground and airborne survey (plus noise).
Show survey values.

\subsection{Comparisons}

Describe the target grid. Show target grid figure.
Describe the parameters used for each source distribution, how do we score each interpolation.
Describe the Harmonica implementation of the EQL.

Show results.

%%%%%%%%%%%%%%%%%%%%%%%%%%%%%%%%%%%%%%%%%%%%%%%%%%%%%%%%%%%%%%%%%%%%%%%%%%%%%%%

\section{Discussion}

%%%%%%%%%%%%%%%%%%%%%%%%%%%%%%%%%%%%%%%%%%%%%%%%%%%%%%%%%%%%%%%%%%%%%%%%%%%%%%%

\section{Conclusions}

%%%%%%%%%%%%%%%%%%%%%%%%%%%%%%%%%%%%%%%%%%%%%%%%%%%%%%%%%%%%%%%%%%%%%%%%%%%%%%%

\section{Acknowledgements}

We are indebted to the developers and maintainers of the open-source software without
which this work would not have been possible.

%%%%%%%%%%%%%%%%%%%%%%%%%%%%%%%%%%%%%%%%%%%%%%%%%%%%%%%%%%%%%%%%%%%%%%%%%%%%%%%

\bibliographystyle{plainnat}
\bibliography{references}

\end{document}
